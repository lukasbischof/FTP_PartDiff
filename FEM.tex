\section{Finite Elements}

\numericsubsection{Variational Problem: Method of Ritz}

Having a two-point boundary problem $-u''(x) = f(x)$ with $v(0)=v(1)=0$ on $[0,1]$, we need to find a function
that minimises the functional $\phi(v) = \int_0^1\sfrac{1}{2}\cdot v'(x)^2-f(x)\cdot v(x)\ dx$

Idea of Ritz: Focusing on a vector subspace $\mathbb{V}^{(n)}\subset \mathbb{V}$ in which we know
that there's a solution and then minimise the functional for all linear combinations:
\begin{align*}
    a_1v_1(x)+\ldots+a_nv_n(x),\quad a_k\in\mathbb{R}
\end{align*}

We have the Ritz matrix
\begin{align*}
    R_{j,k}^{(n)} & = \int_0^1 v_j'(x)\cdot v_k'(x)\ dx \\
    & = v_j'(x)\cdot v_k(x)|_0^1 - \int_0^1 v_j''(x)\cdot v_k(x)\ dx \\
    & = - \int_0^1 v_j''(x)\cdot v_k(x)\ dx\quad\text{\color{gray} (homogeneous b.c.)}
\end{align*}
and the Ritz Vector
\begin{align*}
    r_k^{(n)} = \int_0^1 f(x)\cdot v_k(x)\ dx
\end{align*}
to find the $a$ coefficients: $R^{(n)}\cdot\vec{a} = \vec{r}^{(n)}$.

\subsubsection{Example}

\textbf{Given} Baby Poisson $-u''(x) = f(x)$ with ${\color{blue}f(x) = \pi^2\sin(\pi x)}$ with $u(0) = u(1) = 0$.
We want to compute the coefficient $b$ in the ansatz $\utild(x) = b\cdot(x - 2x^2 + x^3)$.

\textbf{Solution} We compute the (1-dimensional) Ritz matrix $R$ with
${\color{red} v1(x) = x - 2x^2 + x^3}$:
\begin{align*}
    R_{11} = \int_0^1 {\color{red}v_1}'(x)\cdot {\color{red}v_1}'(x)\ dx = \int_0^1 {\color{red}(1-4x+3x^2)}^2\ dx = \sfrac{2}{15}
\end{align*}

and the (1-dimensional) Ritz vector with
\begin{align*}
    r_1=\int_0^1 {\color{blue} f(x)}\cdot {\color{red}v_1(x)}\ dx = \int_0^1 {\color{blue} \pi^2\sin(\pi x)}\cdot{\color{red}(x-2x^2+x^3)}\ dx = \frac{2}{\pi}
\end{align*}

Thus, we get $R_{11}\cdot b = r_1$ and therefore $b=\sfrac{15}{\pi}$.

\numericsubsection{Method of Galerkin}

Weak reformulation: Find a function $u(x)$ (for problem $-u''(x)=f(x)\Rightarrow u''(x)+f(x)=0$) in $\mathbb{V}$ such that for all $v(x)\in\mathbb{V}$ one has
\begin{align*}
    \int_0^1 \left(u''(x) + f(x)\right)\cdot v(x)\ dx = 0
\end{align*}
and then focus on $n$ carefully chosen functions $v_1(x),\ldots,v_n(x)$ and find a function $\utild(x)$ in $\mathbb{V}^{(n)}$,
called ansatz, (that already satisfy the Dirichlet boundary conditions) such that, when substituted for the above integral
\begin{align*}
    \int_0^1 \left(\utild''(x) + f(x)\right)v_k(x)\ dx = 0,\quad {\color{gray} k = 1,\ldots,n}
\end{align*}

The $n$ equations thus turn into $n$ linear equations
\colorbox{shadecolor}{$
    \displaystyle
    \int_0^1 \left(a_1\cdot v_1''(x)+\ldots+a_n\cdot v_n''(x) + f(x)\right)\cdot v_k(x)\ dx = 0
$}

\colorbox{shadecolor}{for $k = 1,\ldots,n$},
giving the Galerkin Matrix $G_{k,j}^{(n)} = \int_0^1 v_j''(x)\cdot v_k(x)\ dx$ and the
Galerkin vector $g_k^{(n)} = \int_0^1 f(x)\cdot v_k(x)\ dx$ and the system $G^{(n)}\cdot\vec{a} + \vec{g}^{(n)} = 0$.
We can also observe that $G^{(n)} = -R^{(n)}$ and $\vec{g}^{(n)} = \vec{r}^{(n)}$.

\subsubsection{Example}

\textbf{Given} The function $u(x)$ on $\Omega = [0,1]$ satisfies Helmholtz's equation
$u''(x) + 17u(x) = 0$ (\emph{= 0 important, move to left side}) with Dirichlet conditions
$u(0) = 0, u(1) = 1$. Determine an approx. function $\utild(x)$ for $u(x)$ with the ansatz
$\utild(x) = x + a_1(x-x^2) + a_2(x^2-x^3)$ (that already satisfies b.c.)

\textbf{Solution}
Given the ansatz
\begin{align*}
    \utild(x) & = x + a_1{\color{blue}(x-x^2)} + a_2{\color{blue}(x^2-x^3)} \\
    \utild''(x) &  = -2a_1 + a_2(2-6x) \\
    {\color{blue} v_1(x)} & {\color{blue} = x-x^2} \\
    {\color{blue}v_2(x)} & {\color{blue}= x^2 - x^3}
\end{align*}
we receive the linear system
\begin{align*}
    & \int_0^1 \left(\utild''(x) + 17\utild(x)\right)\cdot {\color{blue}v_1(x)}\ dx = 0 \\
    & \int_0^1 \left(\utild''(x) + 17\utild(x)\right)\cdot {\color{blue}v_2(x)}\ dx = 0
\end{align*}

which gives
\resizebox{\columnwidth}{!}{$
    \displaystyle
    \int_0^1 \left[-2a_1 + a_2(2-6x) + 17(x + a_1{\color{blue}(x-x^2)} + a_2{\color{blue}(x^2-x^3)})\right]\cdot
    {\color{blue}(x-x^2)}\ dx = 0
$}
\resizebox{\columnwidth}{!}{$
    \displaystyle
    \int_0^1 \left[-2a_1 + a_2(2-6x) + 17(x + a_1{\color{blue}(x-x^2)} + a_2{\color{blue}(x^2-x^3)})\right]\cdot
    {\color{blue}(x^2-x^3)}\ dx = 0
$}

separating by coefficients:
\resizebox{\columnwidth}{!}{$
    \displaystyle
    \int_0^1 \left[a_1(-2+17x-17x^2) + a2(2-6x+17x^2-17x^3)\right]\cdot
    {\color{blue}(x-x^2)}\ dx = 0
$}
\resizebox{\columnwidth}{!}{$
    \displaystyle
    \int_0^1 \left[-2a_1 + a_2(2-6x) + 17(x + a_1{\color{blue}(x-x^2)} + a_2{\color{blue}(x^2-x^3)})\right]\cdot
    {\color{blue}(x^2-x^3)}\ dx = 0
$}

leaves a system in the form $a_1\int_0^1\cdots + a_2\int_0^1\cdots + \int_0^1\cdots = 0$. With the integrals solved, we get:
\begin{align*}
    a_1 \frac{7}{30} + a_2\frac{7}{60} + \frac{17}{12} = 0 \\
    a_1 \frac{7}{60} + a_2\frac{1}{85} + \frac{17}{12} = 0
\end{align*}

and we receive $a_1 \approx -8.45$ and $a_2 \approx 4.76$ and thus
$\utild(x) = x - 8.45 (x-x^2) + 4.76(x^2 - x^3)$.

\numericsubsection{Elliptic}

We have a set of given local basis functions $v_1(x), \ldots, v_n(x)$ that are continuous and piecewise differentiable.
We subdivide $\Omega$ into meshes and assign each nodal point a nodal variable $a_k$.
Then, we use shape functions to represent the PDE on the mesh,
e.g. using triangular functions $l_1(x)=1-x$ and $l_2(x) = x$ to obtain \emph{local} element matrices
\begin{align*}
	E = 
	\begin{bmatrix}
		\int_0^1 l_1^\prime(s)\cdot l_1^\prime(s)\ ds & \int_0^1 l_1^\prime(s)\cdot l_2^\prime(s)\ ds \\
		\int_0^1 l_2^\prime(s)\cdot l_1^\prime(s)\ ds & \int_0^1 l_2^\prime(s)\cdot l_2^\prime(s)\ ds
	\end{bmatrix}
	=
	\begin{bmatrix}
		1 & -1 \\
		-1 & 1
	\end{bmatrix}
\end{align*}

that can then be used to construct the mesh matrix $M=1/h\cdot E$ using mesh size $h$.
Afterwards, the global Ritz matrix can be computed, e.g. for a one-dimensional mesh with 4 nodal points and 2 unknown inner points,
yielding 4 base functions $v_1,v_2,v_3,v_4$:

\begin{align*}
	R^4 = \begin{bmatrix}
		{\color{red} M^{(1)}_{0,0}} & \color{red}{M^{(1)}_{0,1}} & 0 & 0 \\
		{\color{red} M^{(1)}_{1,0}} & {\color{red} M^{(1)}_{1,1}} + {\color{blue} M^{(2)}_{0,0}} & {\color{blue} M^{(2)}_{0,1}} & 0 \\
		0 & {\color{blue} M^{(2)}_{1,0}} & {\color{blue} M^{(2)}_{1,1}} + {\color{green} M^{(3)}_{0,0}} & {\color{green} M^{(3)}_{0,1}} \\
		0 & 0 & {\color{green} M^{(3)}_{1,0}} & {\color{green} M^{(3)}_{1,1}}
	\end{bmatrix}
\end{align*}

The system vector can then be calculated:

\begin{align*}
	\vec{r^4} = \begin{pmatrix}
		\int_0^1 f(x)\cdot v_0(x)\ dx \\
		\vdots \\
		\int_0^1 f(x)\cdot v_3(x)\ dx
	\end{pmatrix}
\end{align*}

Finally, we have obtained the Ritz system: $R^4\cdot\vec{a}=\vec{r^4}$.
That yields the approximation function (as defined by the ansatz): $\tilde{u}(x)=\sum_{i=0}^3 a_i\cdot v_i(x)$
3with $a_0$ and $a_3$ fulfilling the boundary conditions.