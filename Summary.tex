\documentclass[10pt,landscape]{article}
\usepackage{amssymb,amsmath,amsthm,amsfonts}
\usepackage{multicol,multirow}
\usepackage{calc}
\usepackage{ifthen}
\usepackage[landscape]{geometry}
\usepackage[colorlinks=true,citecolor=blue,linkcolor=blue]{hyperref}


\ifthenelse{\lengthtest { \paperwidth = 11in}}
    { \geometry{top=.3in,left=.3in,right=.3in,bottom=.3in} }
	{\ifthenelse{ \lengthtest{ \paperwidth = 297mm}}
		{\geometry{top=1cm,left=1cm,right=1cm,bottom=1cm} }
		{\geometry{top=1cm,left=1cm,right=1cm,bottom=1cm} }
	}
\pagestyle{empty}
\makeatletter
\renewcommand{\section}{\@startsection{section}{1}{0mm}%
                                {-1ex plus -.5ex minus -.2ex}%
                                {0.5ex plus .2ex}%x
                                {\normalfont\large\bfseries}}
\renewcommand{\subsection}{\@startsection{subsection}{2}{0mm}%
                                {-1explus -.5ex minus -.2ex}%
                                {0.5ex plus .2ex}%
                                {\normalfont\normalsize\bfseries}}
\renewcommand{\subsubsection}{\@startsection{subsubsection}{3}{0mm}%
                                {-1ex plus -.5ex minus -.2ex}%
                                {1ex plus .2ex}%
                                {\normalfont\small\bfseries}}
\renewcommand{\arraystretch}{1.3}
\makeatother
\setcounter{secnumdepth}{0}
\setlength{\parindent}{0pt}
\setlength{\parskip}{0pt plus 0.5ex}
% -----------------------------------------------------------------------

\title{Partial Differential Equations}

\begin{document}

\raggedright
\footnotesize

\begin{center}
     \textbf{Partial Differential Equations} \\
\end{center}
\begin{multicols}{3}
\setlength{\premulticols}{1pt}
\setlength{\postmulticols}{1pt}
\setlength{\multicolsep}{1pt}
\setlength{\columnsep}{2pt}

\section{Derivatives}
\begin{tabular}{r|l}
	$f(x)$ & $\frac{df}{dx}$ \\
	\hline
	$\sinh(x)$ & $\cosh(x)$ \\
	$\cosh(x)$ & $\sinh(x)$ \\
	$\mathrm{arcsinh}(x)$ & $1 \div \sqrt{x^2+1}$ \\
	$\mathrm{arccosh}(x)$ & $1 \div \sqrt{x^2 - 1}$ ($1<x$)
\end{tabular}

\section{Integrals}
\begin{tabular}[h]{rl}
	$\int x^n\ dx$ & $= \frac{1}{n+1}x^{n+1} + C$ \\
	$\int \frac{1}{x}\ dx$ & $= \ln |x| + C$ \\
	$\int \frac{1}{ax + b}\ dx$ & = $\frac{1}{a} \ln |ax+b| + C$ \\
	$\int \frac{1}{(x+a)^2}\ dx$ & $= -\frac{1}{x+a} + C$ \\
	$\int \frac{1}{1 + x^2}$ & $= \tan^{-1} x + C$ \\
	$\int \ln ax\ dx$ & $= x\ln ax - x + C$ \\
	$\int e^{ax}\ dx$ & $= \frac{1}{a} e^{ax} + C$ \\
	$\int \sin(ax)\ dx$ & $= -\frac{1}{a}\cos(ax) + C$ \\
	$\int \sin^2(ax)\ dx$ & $= \frac{x}{2}-\frac{\sin(2ax)}{4a} + C$ \\
	$\int x\cos x\ dx$ & $= \cos x + x\sin x + C$ \\
	$\int \sinh(ax)\ dx$ & $= a^{-1}\cosh{ax} + C$ \\
	$\int \cosh(ax)\ dx$ & $= a^{-1}\sinh{ax} + C$ \\
\end{tabular}

\section{Linearity}
Given an equation involving function $u(x), x\in\mathbb{R}$ and its derivatives, 
there is a function $F$ describing the relation:
\begin{equation}
	F\left(
	\begin{matrix}
		x,y,z,p,q,s,t,r,\ldots \\
		\downarrow \text{(corresponding to)}\downarrow \\
		x, y, u, \frac{\partial u}{\partial x}, \frac{\partial u}{\partial y}, \frac{\partial^2 u}{\partial x^2},
		\frac{\partial^2 u}{\partial x\partial y},\frac{\partial^2 u}{\partial y^2}, \ldots
	\end{matrix}
	\right) = 0
\end{equation}
(common variable names $p_i\rightarrow\frac{\partial u}{\partial x_i}$ and 
$t_{ij} \rightarrow\frac{\partial^2 u}{\partial x_i\partial x_j}$)

A PDE is \emph{linear} when function $F$ is linear in $u,p_1,\ldots,p_n,t_{11},\ldots,t_{nn},\ldots.$

A PDF is \emph{quasilinear} when function $F$ is linear in $p_1,\ldots,p_n,t_{11},\ldots,t_{nn},\ldots.$

\section{Domain and Boundary}
\begin{itemize}
	\item Domain $\Omega$ is an open subset of $\mathbb{R}^n$ (meaning all points are interior points)
	\item Boundary has to meet conditions:
	\item{ \emph{Dirichlet boundary conditions}: specify values $u$ on $\partial\Omega$: \\
		$u(x) = f(x)\ \forall x\in\partial\Omega$
	}
	\item{\emph{Neumann boundary conditions}: specify derivatives of $u$ on boundary. 
		Only derivatives orthogonal to the boundary give additional information: \\
		normal derivative: $\frac{\partial u}{\partial n} = g(x)\ \forall x\in\partial\Omega$
	}
\end{itemize}

\section{Quasilinear PDEs}

Given a PDE $F(x,y,u,\frac{\partial u}{\partial x},\frac{\partial u}{\partial y}) = 0$, the PDE defines a relation between the two slopes: $F(x,y,u,m_x,m_y)=0$

\subsection{Characteristics}

The PDE

\begin{equation}
	a(x,y,u)\frac{\partial u}{\partial x}+b(x,y,u)\frac{\partial u}{\partial y} - c(x,y,u) = 0
\end{equation}
can be written in vector notation:
\begin{align*}
{\color{red}
\underbrace{
\begin{pmatrix}
a(x,y,u)\\
b(x,y,u)\\
c(x,y,u)
\end{pmatrix}
}_{\vec t}
}
\cdot
{\color{blue}
\begin{pmatrix}
\frac{\partial u}{\partial x}\\
\frac{\partial u}{\partial y}\\
-1
\end{pmatrix}
}
&= 0
\end{align*}

\end{multicols}

\end{document}