\documentclass[10pt,landscape]{article}
\usepackage{multicol,multirow}
\usepackage{calc}
\usepackage{ifthen}
\usepackage[landscape]{geometry}
\usepackage[colorlinks=true,citecolor=blue,linkcolor=blue]{hyperref}
\usepackage[fleqn]{amsmath}
\usepackage{amssymb,amsthm,amsfonts}
\usepackage{graphicx}
\usepackage{wrapfig}

\ifthenelse{\lengthtest { \paperwidth = 11in}}
    { \geometry{top=.3in,left=.3in,right=.3in,bottom=.3in} }
	{\ifthenelse{ \lengthtest{ \paperwidth = 297mm}}
		{\geometry{top=1cm,left=1cm,right=1cm,bottom=1cm} }
		{\geometry{top=1cm,left=1cm,right=1cm,bottom=1cm} }
	}
\pagestyle{empty}
\makeatletter
\renewcommand{\section}{\@startsection{section}{1}{0mm}%
                                {-1ex plus -.5ex minus -.2ex}%
                                {0.5ex plus .2ex}%x
                                {\normalfont\large\bfseries}}
\renewcommand{\subsection}{\@startsection{subsection}{2}{0mm}%
                                {-1explus -.5ex minus -.2ex}%
                                {0.5ex plus .2ex}%
                                {\normalfont\normalsize\bfseries}}
\renewcommand{\subsubsection}{\@startsection{subsubsection}{3}{0mm}%
                                {-1ex plus -.5ex minus -.2ex}%
                                {1ex plus .2ex}%
                                {\normalfont\small\bfseries}}
\renewcommand{\arraystretch}{1.3}
\makeatother
\setcounter{secnumdepth}{0}
\setlength{\parindent}{0pt}
\setlength{\parskip}{0pt plus 0.5ex}
\setlength{\mathindent}{0pt}
% -----------------------------------------------------------------------

\title{Partial Differential Equations}

\begin{document}

\raggedright
\footnotesize

\begin{center}
     \textbf{Partial Differential Equations} \\
\end{center}
\begin{multicols}{3}
\setlength{\premulticols}{1pt}
\setlength{\postmulticols}{1pt}
\setlength{\multicolsep}{1pt}
\setlength{\columnsep}{2pt}

\section{Domain and Boundary}
\begin{itemize}
	\item Domain $\Omega$ is an open subset of $\mathbb{R}^n$ (meaning all points are interior points)
	\item Boundary has to meet conditions:
	\item{ \emph{Dirichlet boundary conditions}: specify values $u$ on $\partial\Omega$: \\
		$u(x) = f(x)\ \forall x\in\partial\Omega$
	}
	\item{\emph{Neumann boundary conditions}: specify derivatives of $u$ on boundary.
		Only derivatives orthogonal to the boundary give additional information: \\
		normal derivative: $\frac{\partial u}{\partial n} = g(x)\ \forall x\in\partial\Omega$
	}
\end{itemize}

\section{Classification}
\begin{center}
  \makebox[\columnwidth]{\includegraphics[width=0.9\columnwidth]{images/decision-tree}}
\end{center}

\subsection{Linearity}
Given an equation involving function $u(x), x\in\mathbb{R}$ and its derivatives,
there is a function $F$ describing the relation:
\begin{align*}
	F\left(
	\begin{matrix}
		x,y,z,p,q,s,t,r,\ldots \\
		\downarrow \text{(corresponding to)}\downarrow \\
		x, y, u, \frac{\partial u}{\partial x}, \frac{\partial u}{\partial y}, \frac{\partial^2 u}{\partial x^2},
		\frac{\partial^2 u}{\partial x\partial y},\frac{\partial^2 u}{\partial y^2}, \ldots
	\end{matrix}
	\right) = 0
\end{align*}
(common variable names $p_i\rightarrow\frac{\partial u}{\partial x_i}$ and
$t_{ij} \rightarrow\frac{\partial^2 u}{\partial x_i\partial x_j}$)

A PDE is \emph{linear} when function $F$ is linear in $u,p_1,\ldots,p_n,t_{11},\ldots,t_{nn},\ldots.$

A PDF is \emph{quasilinear} when function $F$ is linear in $p_1,\ldots,p_n,t_{11},\ldots,t_{nn},\ldots.$

For example, given the heat equation $u_t=\kappa u_{xx}$, $F$ would be $F(p_1,t_{22})=p_1-\kappa t_{22}$.

\subsection{2\textsuperscript{nd} Order PDEs: Symbol Matrix}

The symbol matrix of the 2\textsuperscript{nd} order partial differential operator

\begin{align*}
	L=\sum_{i,j=1}^{n}a_{ij}(x)\frac{\partial^2}{\partial x_i \partial x_j}+\sum_{i=1}^n b_i(x)\frac{\partial}{\partial x_i}+c(x)
\end{align*}

is the symmetric matrix

\begin{align*}
	A=
	\begin{bmatrix}
		a_{11} & a_{12} & \ldots & a_{1n} \\
		a_{21} & a_{22} & \ldots & a_{2n} \\
		\vdots & \vdots & \ddots & \vdots \\
		a_{n1} & a_{n2} & \ldots & a_{nn}
	\end{bmatrix}
\end{align*}

For example:

\begin{center}
  \makebox[\columnwidth]{\includegraphics[width=0.9\columnwidth]{images/symbol-matrix}}
\end{center}

The type of equation can be inferred by the sign of the eigenvalues $\lambda_1,\ldots,\lambda_n$ of the symbol matrix. 
Calculating the determinant ($\det A=\prod_i \lambda_i$) and trace ($\mathrm{tr}\ A=\sum_i \lambda_i$)
reveals information about the signs of its eigenvalues:

\subsubsection{Two variables of PDE}

\begin{align*}
\det A
\left\{
\begin{matrix}
	>0 & \text{elliptic} \\
	=0 & \text{parabolic} \\
	<0 & \text{hyperbolic}
\end{matrix}
\right.
\end{align*}

\subsubsection{Three variables of PDE}

\begin{align*}
	\mathrm{rank}\ A < 2 & \Rightarrow \text{not classified} \\
	\det A\text{ and }\mathrm{tr}\ A\text{ have different sign} & \Rightarrow\text{hyperbolic} \\
	\det A=0\text{, semidefinite (Cholesky)} & \Rightarrow\text{parabolic} \\
	A\text{ or }-A\text{ positive definite (Cholesky)} & \Rightarrow\text{elliptic} \\
	\text{all other cases} & \Rightarrow\text{hyperbolic}
\end{align*}

\section{Quasilinear PDEs}

\subsection{Characteristics}

\begin{wrapfigure}{r}{0.4\columnwidth}
    \centering
    \includegraphics[width=0.4\columnwidth]{images/quasilinear}
\end{wrapfigure}

The PDE

\begin{align*}
	a(x,y,u)\frac{\partial u}{\partial x}+b(x,y,u)\frac{\partial u}{\partial y} = c(x,y,u)
\end{align*}

can be written in vector notation:

\begin{align*}
  {\color{green}
    \underbrace{
      \begin{pmatrix}
        a(x,y,u) \\
        b(x,y,u) \\
        c(x,y,u)
      \end{pmatrix}
    }_{\vec t}
  }
  \cdot
  {\color{black}
    \underbrace{
      \begin{pmatrix}
        \frac{\partial u}{\partial x}\\
        \frac{\partial u}{\partial y}\\
        -1
      \end{pmatrix}
    }_{\vec{n}}
  }
  &= 0
\end{align*}

where $\color{black}\vec{n}$ is a normal vector and $\color{green}\vec{t}$ is always tangential to the solution surface.
Therefore, we can elaborate a solution algorithm:
\begin{enumerate}
	\item{
		Using the Cauchy initial curve, we formulate
		\begin{align*}
			\vec{v}(s) = \begin{pmatrix}
				v_x({\color{red}s} ) & v_y({\color{red}s}) & v_z({\color{red}s})
			\end{pmatrix}^T
		\end{align*}
		which is a point on the initial curve, parameterised by $s$.
	}
	\item{
		Find characteristic curves as solution of the ODEs
		
		\begin{align*}
			\frac{d}{d{\color{blue}t}}
			\begin{bmatrix}
				x({\color{blue}t}, {\color{red}s}) \\
				y({\color{blue}t}, {\color{red}s}) \\
				z({\color{blue}t}, {\color{red}s})
			\end{bmatrix}
			=
			\begin{bmatrix}
				a(x({\color{blue}t},{\color{red}s}), y({\color{blue}t},{\color{red}s}), z({\color{blue}t}, {\color{red}s})) \\
				b(x({\color{blue}t},{\color{red}s}), y({\color{blue}t},{\color{red}s}), z({\color{blue}t}, {\color{red}s})) \\
				c(x({\color{blue}t},{\color{red}s}), y({\color{blue}t},{\color{red}s}), z({\color{blue}t}, {\color{red}s}))
			\end{bmatrix}
		\end{align*}
		with
		\begin{align*}
			\begin{bmatrix}
				x({\color{blue}0}, {\color{red}s}) \\
				y({\color{blue}0}, {\color{red}s}) \\
				z({\color{blue}0}, {\color{red}s})
			\end{bmatrix}
			=
			\vec{v}({\color{red}s})
			=
			\begin{bmatrix}
				v_x({\color{red}s} ) \\
				v_y({\color{red}s}) \\
				v_z({\color{red}s})
			\end{bmatrix}
		\end{align*}
	}
	\item{
		Eliminate the variables {\color{blue}t} and {\color{red}s} and condense solution into a function $u(x, y)$:
		\begin{align*}
			\left.
			\begin{matrix}
				x=x({\color{blue}t},{\color{red}s}) \\
				y=y({\color{blue}t},{\color{red}s}) \\
				u=z({\color{blue}t},{\color{red}s})
			\end{matrix}
			\ \ \right\}
			\ u=u(x,y)
		\end{align*}
	}
\end{enumerate}

\section{Numerical Methods}
\subsection{Discretisation of Operators}
\begin{align*}
	\frac{\partial g}{\partial x}
	\approx
	\frac{g(x + \Delta x) - g(x)}{\Delta x}
\end{align*}
\begin{align*}
	\frac{\partial^2 g}{\partial x^2}
	\approx
	\frac{g(x + \Delta x) - 2\cdot g(x) + g(x - \Delta x)}{\Delta x^2}
\end{align*}

($\Delta$ referring to step size)
\\[1em]
\textbf{Discrete Laplace-Operator / Five-Point-Star Operator}
\\
Setting $h = \Delta x = \Delta y$, then $\nabla^2u$ is
\resizebox{\columnwidth}{!}{$
    \displaystyle
	\frac{1}{h^{2}}\left(
	\underbrace{u(x+h,y)}_{\text{East}}
	+ \underbrace{u(x,y+h)}_{\text{North}}
	+ \underbrace{u(x-h,y)}_{\text{West}}
	+ \underbrace{u(x,y-h)}_{\text{South}}
	- \underbrace{4u(x,y)}_\text{Center}
	\right)
$}


\newpage

\section{Cheat Sheet}
\subsection{Roots}

\begin{align*}
	\sqrt[n]{a}\cdot\sqrt[n]{b} & = \sqrt[n]{a\cdot b} \\
	\frac{\sqrt[n]{a}}{\sqrt[n]{b}} & = \sqrt[n]{\frac{a}{b}} \\
	(\sqrt[n]{a})^m & = \sqrt[n]{a^m} \\
	\sqrt[m]{\sqrt[n]{a}} & = \sqrt[m\cdot n]{a}
\end{align*}


\subsection{Logarithm}

\begin{align*}
	\log_n(a\cdot b) & = \log_n(a) + \log_N(b) \\
	\log_n(a\div b) & = \log_n(a) - \log_N(b) \\
	\log_n(a^b) & = b \cdot \log_n(a)
\end{align*}

\subsection{Trigonometry}

\begin{align*}
	\tan\theta & = \frac{\sin\theta}{\cos\theta} \\
	\sin -\theta & = -\sin\theta\text{ (cos same)} \\
	\sin 2\theta & = 2\sin\theta\cos\theta \\
	\cos 2\theta & = 2\cos^2\theta - \sin^2\theta = 2\cos^2\theta - 1 = 1 - 2\sin^2\theta \\
	\sin(\alpha \pm \beta) & = \sin\alpha\cos\beta\pm\cos\alpha\sin\beta \\
	\cos(\alpha\pm\beta) & = \cos\alpha\cos\beta \mp \sin\alpha\sin\beta
\end{align*}

\subsection{Derivatives}
\begin{tabular}{r|l}
	$f(x)$ & $\frac{df}{dx}$ \\
	\hline
	$\sinh(x)$ & $\cosh(x)$ \\
	$\cosh(x)$ & $\sinh(x)$ \\
	$\mathrm{arcsinh}(x)$ & $1 \div \sqrt{x^2+1}$ \\
	$\mathrm{arccosh}(x)$ & $1 \div \sqrt{x^2 - 1}$ ($1<x$) \\
	$\tan(x)$ & $\cos^{-2}(x)$ \\
	$\log(x)$ & $x^{-1}$
\end{tabular}

\subsection{Integrals}
\begin{tabular}[h]{rl}
	$\int x^n\ dx$ & $= \frac{1}{n+1}x^{n+1} + C$ \\
	$\int \frac{1}{x}\ dx$ & $= \ln |x| + C$ \\
	$\int \frac{1}{ax + b}\ dx$ & = $\frac{1}{a} \ln |ax+b| + C$ \\
	$\int \frac{1}{(x+a)^2}\ dx$ & $= -\frac{1}{x+a} + C$ \\
	$\int \frac{1}{1 + x^2}$ & $= \tan^{-1} x + C$ \\
	$\int \ln ax\ dx$ & $= x\ln ax - x + C$ \\
	$\int e^{ax}\ dx$ & $= \frac{1}{a} e^{ax} + C$ \\
	$\int \sin(ax)\ dx$ & $= -\frac{1}{a}\cos(ax) + C$ \\
	$\int \sin^2(ax)\ dx$ & $= \frac{x}{2}-\frac{\sin(2ax)}{4a} + C$ \\
	$\int x\cos x\ dx$ & $= \cos x + x\sin x + C$ \\
	$\int \sinh(ax)\ dx$ & $= a^{-1}\cosh{ax} + C$ \\
	$\int \cosh(ax)\ dx$ & $= a^{-1}\sinh{ax} + C$ \\
\end{tabular}

\subsection{Integration Techniques}
\subsubsection{Integration by Parts}
\begin{equation*}
	\int_a^b u(x)v'(x)\ dx = \left[ u(x)v(x) \right]_a^b-\int_a^bu'(x)v(x)\ dx
\end{equation*}

Or, with $u=u(x)$, $du=u'(x)\ dx$, $v=v(x)$ and $dv=v'(x)\ dx$:
\begin{equation*}
	\int u\ dv=uv - \int v\ du
\end{equation*}

\subsubsection{Substitution}
\begin{equation*}
	\int_a^b f(g(x))\cdot g'(x)\ dx = \int_{g(a)}^{g(b)}f(u)\ du
\end{equation*}

\subsubsection{Leibniz Integral Rule}
\begin{multline*}
	\frac{d}{dx}\left(\int_{a(x)}^{b(x)}f(x,t)\ dt\right)
	=
	\\
	f(x,b(x))\cdot\frac{d}{dx}b(x)
	-f(x,a(x))\cdot\frac{d}{dx}a(x)
	+\int_{a(x)}^{b(x)}\frac{\partial}{\partial x}f(x,t)\ dt
\end{multline*}
Special case where $a(x)=a=\mathrm{const.}$ and $b(x)=b=\mathrm{const.}$:
\begin{equation*}
	\frac{d}{dx}\left(\int_a^b f(x,t)\ dt\right)
	=\int_a^b\frac{\partial}{\partial x}f(x,t)\ dt
\end{equation*}

\subsection{Particular Solutions to Simple ODEs}

\begin{tabular}[h]{rcl}
	$f'(x)=\frac{c}{x}f(x)$ & $\Rightarrow$ & $f(x)=k_1y^c$ \\
	$f'(x)=x\cdot f(x)$ & $\Rightarrow$ & $f(x)=k_1e^{cx}$ \\
	$f''(x) = c\cdot f(x)$ & $\Rightarrow$ & $f(x) = k_1e^{\sqrt{c}x}+k_2e^{-\sqrt{c}x}$ \\
	$f''(x) = -c\cdot f(x)$ & $\Rightarrow$ & $f(x)=k_1\sin(\sqrt{c}x)+k_2\cos(\sqrt{c}x)$ \\
	$f'(x)+af(x) = b$ & $\Rightarrow$ & $f(x) = \left(f(0)-\frac{b}{a}\right)e^{-ax}+\frac{b}{a}$
\end{tabular}

\end{multicols}

\end{document}
