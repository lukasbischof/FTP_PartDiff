\section{Numerical Methods}
\subsection{Discretisation of Operators}

\begin{align*}
	\frac{\partial g}{\partial x}
	\approx
	\frac{g(x + \Delta x) - g(x)}{\Delta x}
\end{align*}

\begin{align*}
	\frac{\partial^2 g}{\partial x^2}
	\approx
	\frac{g(x + \Delta x) - 2\cdot g(x) + g(x - \Delta x)}{\Delta x^2}
\end{align*}

($\Delta$ referring to step size)
\subsection{Finite Elements}

We have a set of given local basis functions $v_1(x), \ldots, v_n(x)$ that are continuous and piecewise differentiable.
We subdivide $\Omega$ into meshes and assign each nodal point a nodal variable $a_k$.
Then, we use shape functions to represent the PDE on the mesh,
e.g. using triangular functions $l_1(x)=1-x$ and $l_2(x) = x$ to obtain \emph{local} element matrices
\begin{align*}
	E = 
	\begin{bmatrix}
		\int_0^1 l_1^\prime(s)\cdot l_1^\prime(s)\ ds & \int_0^1 l_1^\prime(s)\cdot l_2^\prime(s)\ ds \\
		\int_0^1 l_2^\prime(s)\cdot l_1^\prime(s)\ ds & \int_0^1 l_2^\prime(s)\cdot l_2^\prime(s)\ ds
	\end{bmatrix}
	=
	\begin{bmatrix}
		1 & -1 \\
		-1 & 1
	\end{bmatrix}
\end{align*}

that can then be used to construct the mesh matrix $M=1/h\cdot E$ using mesh size $h$.
Afterwards, the global Ritz matrix can be computed, e.g. for a one-dimensional mesh with 4 nodal points and 2 unknown inner points,
yielding 4 base functions $v_1,v_2,v_3,v_4$:

\begin{align*}
	R^4 = \begin{bmatrix}
		{\color{red} M^{(1)}_{0,0}} & \color{red}{M^{(1)}_{0,1}} & 0 & 0 \\
		{\color{red} M^{(1)}_{1,0}} & {\color{red} M^{(1)}_{1,1}} + {\color{blue} M^{(2)}_{0,0}} & {\color{blue} M^{(2)}_{0,1}} & 0 \\
		0 & {\color{blue} M^{(2)}_{1,0}} & {\color{blue} M^{(2)}_{1,1}} + {\color{green} M^{(3)}_{0,0}} & {\color{green} M^{(3)}_{0,1}} \\
		0 & 0 & {\color{green} M^{(3)}_{1,0}} & {\color{green} M^{(3)}_{1,1}}
	\end{bmatrix}
\end{align*}

The system vector can then be calculated:

\begin{align*}
	\vec{r^4} = \begin{pmatrix}
		\int_0^1 f(x)\cdot v_0(x)\ dx \\
		\vdots \\
		\int_0^1 f(x)\cdot v_3(x)\ dx
	\end{pmatrix}
\end{align*}

Finally, we have obtained the Ritz system: $R^4\cdot\vec{a}=\vec{r^4}$.
That yields the approximation function (as defined by the ansatz): $\tilde{u}(x)=\sum_{i=0}^3 a_i\cdot v_i(x)$
3with $a_0$ and $a_3$ fulfilling the boundary conditions.