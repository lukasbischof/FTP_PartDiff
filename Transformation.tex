\subsection{Transformation}\label{subsec:transformation}
\begin{enumerate}
    \item{
        Transform equation: derivatives turn into algebraic expressions:
        \begin{align*}
            \left(
            \mathcal{L}\frac{\partial u(t,y)}{\partial t}
            \right)(s) =
            s
            \underbrace{(\mathcal{L}f)(s, y)}_{=y_s(y)}
            -u(0,y)
        \end{align*}
    }
    \item{
        Transform y-boundary conditions: gives boundary conditions for
        \begin{align*}
            u(t,0) & =g(t) \\
            y_s(0)=(\mathcal{L}u(t,0))(s) & = (\mathcal{L}g)(s)
        \end{align*}
    }
    \item Solve PDE with fewer derivatives, ODEs
    \item Inverse transform
\end{enumerate}

\subsubsection{Laplace Transform}

(Only works on linear equations.)
The Laplace transform of a function $f:\mathbb{R}^+\to \mathbb{R}$ is the function

\begin{align*}
    \mathcal{L}f : \mathbb{R}^+\to \mathbb{R} : s \rightarrowtail \mathcal{L}f(s) = \int_0^\infty f(t)e^{-st}\ dt
\end{align*}

It is linear:

\begin{align*}
	\mathcal{L}(\alpha f+\beta f)=\alpha\mathcal{L}f+\beta\mathcal{L}f
\end{align*}

Example transformations:
\begin{align*}
	\begin{matrix}
		\text{Constant: } & \text{Exponential: } & \text{Derivative: } \\
		f(t)=c & f(t)=e^{-ct} & f(t)=g^{(n)}(t) \\
		(\mathcal{L}f)(s) = \frac{c}{s} & (\mathcal{L}f)(s) = \frac{1}{c+s} & (\mathcal{L}f^{(n)})(s)= \\
		& & -f^{(n-1)}(0)+s\left(\mathcal{L}f^{(n-1)}\right)(s) \\
		& & \text{(removes t-derivatives: }\frac{\partial}{\partial t}\rightarrow s\text{)}
	\end{matrix}
\end{align*}

\subsubsection{Fourier Transform}

For $f : \mathbb{R}\to\mathbb{C} : x \rightarrowtail f(x)$ the Fourier transform of $f$ is defined as

\begin{align*}
    \mathcal{F}f = \hat{f} : \mathbb{R}\to\mathbb{C} : k\rightarrowtail
    \frac{1}{\sqrt{2\pi}}\int_{-\infty}^{\infty}\hat{f}(k)e^{-ikx}\ dx
\end{align*}

It turns the derivative $\frac{\partial}{\partial x}$ into a multiplication by $-ik$ (second derivatives are reduced to $i^2=-1$):
\begin{align*}
	(\mathcal{F}f^{(n)})(k) = (ik)^n\mathcal{F}f(k)
\end{align*}

The function $f$ can be recovered from $\hat{f}$ by
\begin{align*}
    f(x)=(\mathcal{F}^{-1}\hat{f})(x)
    = \frac{1}{\sqrt{2\pi}}\int_{-\infty}^\infty \hat{f}(k) e^{ikx}\ dk
\end{align*}
