\section{Hyperbolic PDEs}
\subsection{D'Alembert Solution of Wave Equation}
The wave equation can be factorized:
\begin{align*}
    0 = \frac{\partial^2 u}{\partial t^2} - a^2 \frac{\partial^2 u}{\partial x^2}
    = \left(
    \frac{\partial}{\partial t}-a\frac{\partial}{\partial x}
    \right)
    \left(
    \frac{\partial}{\partial t}-a\frac{\partial}{\partial x}
    \right)
    u
\end{align*}
giving two quasilinear PDEs of which the superposition is a solution:

$u(t,x) = u_+(x-at)+u_-(x+at)$

\subsection{Strip and Characteristics}

Consider the hyperbolic equation
\colorbox{shadecolor}{$
    \displaystyle
    a\frac{\partial^{2}u}{\partial x^{2}}+2b\frac{\partial^{2}u}{\partial x\partial y}+c\frac{\partial^{2}u}{\partial y^{2}}=g-d\frac{\partial u}{\partial x}+e\frac{\partial u}{\partial y}+f u=h.
$}

A strip then is a curve $(x(s), y(s), u(x))$ together with the slopes of the tangent planes
$p(s) = \frac{\partial u}{\partial x}$ and $q(s) = \frac{\partial u}{\partial y}$.

The second partial derivatives are then determined by the linear system
\begin{align*}
    a\frac{\partial^{2}u}{\partial x^{2}}+2b\frac{\partial^{2}u}{\partial x\partial y}
    + c\frac{\partial^{2}u}{\partial y^{2}} & = h(t)\,=g-d p(t)-e q(t)-f u \\
    \dot{x}(t)\frac{\partial^{2}u}{\partial x^{2}}+\dot{y}(t)\frac{\partial^{2}u}{\partial x\partial y} & = \dot{p}(t) \\
    \dot{x}(t)\frac{\partial^{2}u}{\partial x\partial y} + \dot{y}(t)\frac{\partial^{2}u}{\partial y^2} & = \dot{q}(t)
\end{align*}

The \emph{characteristics} of a differential equation are the curves $t\mapsto(x(t),y(t))$ for which the initial data
does not determine the second partial derivatives uniquely and thus, determinant of the linear system is zero
\begin{align*}
    \det
    \begin{bmatrix}
        a & 2b & c \\
        \dot{x}(t) & \dot{y}(t) & 0 \\
        0 & \dot{x}(t) & \dot{y}(t)
    \end{bmatrix}
    = 0
\end{align*}

The determinant is constructible from the \emph{symbol matrix $A$} and we thus receive a differential equation for the characteristics:
\colorbox{shadecolor}{$
    \displaystyle
    A = \begin{bmatrix}
        a & b \\
        b & c
    \end{bmatrix}
    \quad\Rightarrow\quad
    a\dot{y}(t)^2 - 2b\dot{x}\dot{y}(t) + c\dot{x}(t)^2 = 0
$}

(note: $t$ is the mapping variable of the curve, if the PDE uses time (e.g. wave) $s$ may be used)

\textbf{Consequence:} A solution $u(t,x)$ of the wave equation determines a strip for each value $t$:
\resizebox{\columnwidth}{!}{$
    x(s) = s\quad t(s) = t\quad u(s) = u(t,s)\quad p(s) = \frac{\partial u}{\partial x}(s,t)\quad q(s) = \frac{\partial u}{\partial t}(s,t)
$}

with an initial strip

$x(s) = s\quad t(s) = 0\quad u(s) = f(s)\quad p(s) = f'(s)\quad q(s)=g(s)$

the Cauchy-Problem therefore has to be formulated in terms of the strip.